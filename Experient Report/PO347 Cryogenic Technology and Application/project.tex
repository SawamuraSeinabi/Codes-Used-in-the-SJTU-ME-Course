\documentclass[UTF8,a4paper]{ctexart}
\usepackage[left=2.50cm, right=2.50cm, top=2.50cm, bottom=2.50cm]{geometry} %页边距
\CTEXsetup[format={\Large\bfseries}]{section} %设置章标题居左


%%%%%%%%%%%%%%%%%%%%%%%
% -- text font --
% compile using Xelatex
%%%%%%%%%%%%%%%%%%%%%%%
% -- 中文字体 --
%\setmainfont{Microsoft YaHei}  % 微软雅黑
%\setmainfont{YouYuan}  % 幼圆    
%\setmainfont{NSimSun}  % 新宋体
%\setmainfont{KaiTi}    % 楷体
%\setmainfont{SimSun}   % 宋体
%\setmainfont{SimHei}   % 黑体
% -- 英文字体 --
%\usepackage{times}
%\usepackage{mathpazo}
%\usepackage{fourier}
%\usepackage{charter}
\usepackage{helvet}


\usepackage{amsmath, amsfonts, amssymb} % math equations, symbols
\usepackage[english]{babel}
\usepackage{color}      % color content
\usepackage{graphicx}   % import figures
\usepackage{url}        % hyperlinks
\usepackage{bm}         % bold type for equations
\usepackage{multirow}
\usepackage{booktabs}
\usepackage{epstopdf}
\usepackage{epsfig}
\usepackage{algorithm}
\usepackage{algorithmic}
\usepackage{graphicx}
\usepackage{subfigure}
\usepackage{changepage}
\usepackage[hidelinks]{hyperref}
\makeatletter
\def\UrlAlphabet{%
	\do\a\do\b\do\c\do\d\do\e\do\f\do\g\do\h\do\i\do\j%
	\do\k\do\l\do\m\do\n\do\o\do\p\do\q\do\r\do\s\do\t%
	\do\u\do\v\do\w\do\x\do\y\do\z\do\A\do\B\do\C\do\D%
	\do\E\do\F\do\G\do\H\do\I\do\J\do\K\do\L\do\M\do\N%
	\do\O\do\P\do\Q\do\R\do\S\do\T\do\U\do\V\do\W\do\X%
	\do\Y\do\Z}
\def\UrlDigits{\do\1\do\2\do\3\do\4\do\5\do\6\do\7\do\8\do\9\do\0}
\g@addto@macro{\UrlBreaks}{\UrlOrds}
\g@addto@macro{\UrlBreaks}{\UrlAlphabet}
\g@addto@macro{\UrlBreaks}{\UrlDigits}
\makeatother
\urlstyle{same}

\usepackage{abstract}
\usepackage{titlesec}
\usepackage{caption}
\usepackage{ctex}
\linespread{2}

\usepackage{listings}
\usepackage{xcolor}
\lstset{
	numbers=left, 
	numberstyle= \tiny, 
	keywordstyle= \color{ blue!70},
	commentstyle= \color{red!50!green!50!blue!50}, 
	frame=shadowbox, % 阴影效果
	rulesepcolor= \color{ red!20!green!20!blue!20} ,
	escapeinside=``, % 英文分号中可写入中文
	xleftmargin=0.5em,xrightmargin=0.5em, aboveskip=1em,
	framexleftmargin=1em
}

\renewcommand{\algorithmicrequire}{ \textbf{Input:}}     % use Input in the format of Algorithm  
\renewcommand{\algorithmicensure}{ \textbf{Initialize:}} % use Initialize in the format of Algorithm  
\renewcommand{\algorithmicreturn}{ \textbf{Output:}}     % use Output in the format of Algorithm  
\newcommand{\upcite}[1]{\textsuperscript{\textsuperscript{\cite{#1}}}}
\renewcommand{\abstractnamefont}{\Large\bfseries}


\usepackage{fancyhdr} %设置页眉、页脚
%\pagestyle{fancy}
\lhead{}
\chead{}
%\rhead{\includegraphics[width=1.2cm]{fig/ZJU_BLUE.eps}}
\lfoot{}
\cfoot{}
\rfoot{}


%%%%%%%%%%%%%%%%%%%%%%%
%  设置水印
%%%%%%%%%%%%%%%%%%%%%%%
%\usepackage{draftwatermark}         % 所有页加水印
%\usepackage[firstpage]{draftwatermark} % 只有第一页加水印
% \SetWatermarkText{Water-Mark}           % 设置水印内容
% \SetWatermarkText{\includegraphics{fig/ZJDX-WaterMark.eps}}         % 设置水印logo
% \SetWatermarkLightness{0.9}             % 设置水印透明度 0-1
% \SetWatermarkScale{1}                   % 设置水印大小 0-1    



\title{\textbf{小型车载液氢储罐设计与仿真计算}}
\author{}
\date{\today}

\begin{document}
	\zihao{4}
	\titleformat{\section}{\huge\bfseries}{\thesection}{1em}{}
	\titleformat{\subsection}{\Large\bfseries}{\thesubsection}{1em}{}
	\titleformat{\subsubsection}{\Large\bfseries}{\thesubsubsection}{1em}{}
	\titleformat{\subsubsection}{\Large\bfseries}{\thesubsubsection}{1em}{}
	\maketitle
	
	\newpage
	\tableofcontents
	
	\newpage
	\clearpage
	\begin{abstract}
		\zihao{4}
		\phantomsection
		\addcontentsline{toc}{section}{Abstract}
		车载液氢气瓶是作为液氢储存的核心部件。本文主要针对车载液氢气瓶的结构、材料、绝热性能以及车辆处于不同工况下气瓶内筒应力应变变化规律进行分析研究。针对各类气瓶,选择合适的内筒材料、支撑件材料,并设计其相应的结构,最终选择内筒直径300mm,外筒直径400mm,体积约为60L。参照国家标准设计内筒长度为730mm,壁厚为5mm。此外,本文介绍了车载液氢气瓶常用绝热材料、绝热类型以及研究现状,分析气瓶内不同部位的漏热情况,进行漏热计算,得到液氢的日蒸发率为6.53\%。最后,本文利用ANSYS对气瓶内筒进行漏热计算,计算其在工作压力以及设计压力下气瓶应力应变的变化规律。
		
	\end{abstract}
	
	\newpage
	\section{绪论}
	\subsection{背景介绍}
	伴随着科技的不断向前进步,包括着汽车这一产业的技术也在不断发展,人们也越来越热衷于将汽车作为自己出行的交通工具之一。基于此,越来越多的企业投入到汽车的研发和生产之中,全球汽车的总量也成几何倍数增长中。但是随着汽车数量的激增,汽车也带来了许多的问题,比如道路堵塞,目前各大城市都出现了交通堵塞,浪费了人们太多的时间;资源消耗,大量的汽车使用耗费了全球大量石油等资源的消耗;同时带来最重大也是人们现在最头痛的问题:地球环境变得越来越差。据统计每天全球汽车排出的尾气占到全球日污染气体总量的百分之十五左右,这是一个很庞大的数字。所以在人们对汽车技术的研究中,研制出资源消耗少,同时对环境污染小的新型节能环保汽车已然成为了人们研究的重点领域。
	
	环境污染、全球变暖、资源匮乏等问题,迫使人们加大力度对新能源进行开发。探索新燃料和更好的燃料是应用研究中一个非常活跃的领域。氢因其资源丰富和无污染,已被认为是一种很有发展前景的汽车燃料。
	氢作为内燃机燃料的研究始于19 世纪中期,当时曾设计过燃氢的活塞式内燃机,20世纪30年代,美国和德国曾生产了1000多辆以氢或掺氢汽油为燃料的汽车。但二战结束后,因为石油资源的广泛应用,氢动力车的研究曾一度停滞不前。到了70年代,由于石油危机以及人们对环境污染的关心,日本、德国、美国先后实施了各自的氢能发展计划,再度激起了人们对氢动力车的研究热情。
	
	然而,氢如果作为内燃机燃料,燃烧后会产生少量的$NO_{x}$,因此以这种方式利用氢能,氢并不是真正的清洁能源。新近发展出来的先进燃料电池技术和催化燃氢技术,排放物中只有水,因此由这类燃料电池组成的氢动力车可以实现真正的“零排放”,是最理想的交通工具,充分体现了氢能的优越性。
	
	无论氢动力车采用内燃机直接燃烧氢气产生动力,还是采用燃料电池将氢能转化为电能,用电能驱动汽车,都离不开对随车贮氢技术的研究,随车贮氢量已成为目前氢动力车的主要技术问题。美国1992年氢能投资中,50\%用于贮氢,30\%用于制氢,20\%用于氢的应用研究,可见贮氢技术的研究已成为当今研究热点。
	
	2020年9月22日,国家主席习近平在第七十五届联合国大会一般性辩论上发表重要讲话。习近平在讲话中指出,这场疫情启示我们人类需要一场自我革命,加快形成绿色发展方式和生活方式,建设生态文明和美丽地球。人类不能再忽视大自然一次又一次的警告,沿着只讲索取不讲投入、只讲发展不讲保护、只讲利用不讲修复的老路走下去。应对气候变化《巴黎协定》代表了全球绿色低碳转型的大方向,是保护地球家园需要采取的最低限度行动,各国必须迈出决定性步伐。\textbf{中国将提高国家自主贡献力度,采取更加有力的政策和措施,二氧化碳排放力争于2030年前达到峰值,努力争取2060年前实现碳中和。各国要树立创新、协调、绿色、开放、共享的新发展理念,抓住新一轮科技革命和产业变革的历史性机遇,推动疫情后世界经济“绿色复苏”,汇聚起可持续发展的强大合力。}
	
	习近平也宣布,中国将设立联合国全球地理信息知识与创新中心和可持续发展大数据国际研究中心,为落实《联合国2030年可持续发展议程》提供新助力。
	
	同年12月中旬举办的中央经济工作会议确定,明年要抓好八项重点任务,其中包括做好碳达峰、碳中和工作。生态环境部正制定2030年前碳排放达峰行动计划、行动方案,支持有条件的地区率先达到碳排放峰值。在做好碳达峰、碳中和工作方面,从根本和源头上作出部署,明确加快调整优化产业结构、能源结构,以及大力发展新能源,继续打好污染防治攻坚战等。要加快调整优化产业结构、能源结构,推动煤炭消费尽早达峰,大力发展新能源,加快建设全国用能权、碳排放权交易市场,完善能源消费双控制度。要继续打好污染防治攻坚战,实现减污降碳协同效应。开展大规模国土绿化行动,提升生态系统碳汇能力。
	
	正因如此,碳中和、碳达峰将成为我国“十四五”污染防治攻坚战的主攻目标。也因此,开发不含碳的新能源刻不容缓。
	\subsection{氢能的优点}
	氢能因其具有以下四大特点,而被视为新世纪重要二次能源:
	
	1、氢的高含能特性。氢的热值为120$MJ/kg$,是酒精、焦炭热值的4倍,汽油热值的2.6倍,天然气热值的1.5倍;
	
	2、利用氢气可以提高能源转化率。由卡诺循环理论计算,化石原料燃烧发热通过内燃机转换为有效功的效率在2000℃时为70\%,1000℃时为56\%,再将机械能转换为电能时,效率会进一步下降。而氢能通过燃料电池可以直接转换为电能,提高了转换效率,即便直接燃烧氢气,因为其温度可达2000℃,其热机效率仍很高。
	
	3、碳的零排放。氢在燃烧或者在燃料电池产生电能的反应后,不会产生导致全球变暖的$CO_{2}$气体,而只有无污染的水,可以实现良性循环。
	
	4、氢气可以作为一种高密度能源存储的载体。与其他储能形式相比,氢气可以通过气相、液相和固相,提供一种大规模、高密度、方便移动的存储能量途径。除此之外,氢能还具有资源丰富、来源多样性等特点,虽然在地球上自然条件下形成的氢气不多,但是在水中含有丰富的氢元素,天然气和石油当中也存在一定量的氢,因此,氢气是一种取之不尽的洁净能源,成为各国能源战略转移和研究的重点,有着很大的发展潜力。
	\subsection{储氢与燃料电池}
	氢能的利用离不开燃料电池,氢燃料电池应用最多的领域是在交通运输领域,其中燃料电池汽车具有绿色环保和续驶里程长等优势,被视为绿色低碳交通的有效解决方案之一,随着氢燃料电池汽车技术的逐渐成熟和商业化应用扩大,以及氢气生产储运、加氢站和燃料电池研发等产业的发展,氢燃料电池汽车被视为燃料电池众多应用中最先实现产业化的领域。而且国际氢能源委员会预计到2050年,氢能源需求将是目前10倍,到2030年,全球燃料电池乘用车将达到1000万辆至1500万辆,由此可见氢燃料电池汽车未来市场之大。
	
	除了燃料电池自身结构和性能的研究以外,车载储氢供氢系统是氢燃料电池汽车的.另一个重要研究方向。就当前技术而言,高压气态储氢和低温液态储氢是众多储氢方式中,可实际应用于氢燃料电池汽车的两种主要方式。
	
	高压储氢是采用刚性的气瓶,利用高压将氢气压缩存储,目前高压气瓶的储氢压力较多采用35MPa,为了提高气瓶储存氢气的容量,美国Quantum公司在2002年之后研制出工作压力70MPa的高压气瓶,之后,日本丰田公司也研制出70MPa高压气瓶,并在其燃料电池汽车MIRAI上使用。近年来,国内一些企业,如北京科泰克、中国中氢、中材等企业,也相继研制出70MPa高压气瓶,但是由于尚无国家标准,70MPa气瓶未应用于车载储氢系统上,但是随着车载储氢标准的完善和续航里程提高的要求,70MPa气瓶将逐步取代35MPa气瓶。采用高压气态储氢,尽管将压力从35MPa提升至70MPa,气瓶的质量储氢密度也仅仅从3wt\%提升至4.5wt\%,储氢密度仍然不高。
	
	低温液态储氢就是采用真空绝热的容器存储液氢,与高压储氢相比,液态储氢具有如下优点:
	
	1.储氢质量和储氢密度大。液态储氢系统的质量储氢密度约为7wt左右,一个500L液氢储罐,储氢量相当于10个140L车载35MPa高压气瓶的储氢量,液态储氢不仅大大提高了燃料电池汽车的储氢质量,提高续航能力,而且还占据空间优势,在储氢量相同情况下,液态储氢系统占用的空间约为35MPa气态系统的三分之一。
	
	2.安全性能高。液态储氢系统无需高压,危险性大大降低。
	
	3.存在大量冷量可以利用,液氢汽化过程或低温氢气升温过程,会产生大量的冷量,这些冷量可以应用于空调制冷系统中。
	
	从文献\upcite{ref1}中可以看出,液氢的储能密度远高于高压气氢和金属氢化物。
	
	在加注方面,氢气在管道内流动,当流速大时,与管道壁面摩擦增强,特别是管道内含有铁锈等杂质时,容易形成静电火花,引发事故。为了确保安全,对于使用高压氢气系统的充装,管道流速必须很低。但是对于高压工况,要精确控制气体流速是比较困难的。由于气态氢能量密度低,加注速度又有限制,因此加注时间较长。
	
	液氢加注站使用常压低温液态氢加注,加注压力低,由于能量致密,加注时间短,一个120~150L的液氢储罐,加注时间约5~10min,如果采用自动化程度较高的加注站加注,仅需31min。
	
	金属氢化物的加注时间很长,加满一个5kg的储氢瓶约需1h。虽然提高充气压力可加快吸附,但大量的吸附热会导致金属氢化物的温度过高,以至烧坏。此外,氢化物储氢对氢气的纯度有较高要求,因为杂质在氢化物中的吸附是不可逆的,在多次充放氢后,性能就会衰退。因此,除气瓶本身的使用寿命外,金属氢化物也有耐久性的问题。
	
	\begin{figure}[H]
		\centering
		\includegraphics[width=0.6\linewidth]{biao1-1}
		\caption*{表1-1 不同储氢方式对应的燃料和燃料箱质量}
		\label{fig:biao1-1}
	\end{figure}
	通过以上对比分析,尽管现阶段高压储氢技术相对成熟,液态储氢技术面临易挥发、商业化难度大、运行安全低等关键技术问题,但是,考虑到液态储氢的储氢质量大等优势,并且现在的汽车用的也是液态燃料,液态储氢仍是未来较为理想的车载储氢技术途径。
	\section{气瓶结构研究}
	\subsection{气瓶综述}
	\subsubsection{气瓶总体结构}
	液氢在$1\times10^{5}Pa$,$20K$的条件下液化,整体处于一个低温保冷的情况,因此必须选用绝热性能非常好的容器进行储藏,有效避免漏热的情况出现。对于整体而言,液氢气瓶一般为双层结构,分为内筒、外筒以及中间绝热层三个部分 \upcite{ref2}。
	
	首先是内筒部分。内层中装有20K的液氢作为车辆燃料,由于其与低温液体直接接触,因此其需要满足较好的刚度、耐低温性以及气密性。
	
	其次是外筒部分。外筒主要起到保护内部构件的作用,通过支撑物与内胆连接。为避免因为连接件导致的额外散热,且有较好的强度和韧性,连接件通常使用玻璃纤维,具有较好的绝热性能。
	
	最后是中间绝热层部分。对于绝热层而言,其主要目的是隔绝内外筒之间的热辐射以及热对流造成的漏热,因此其中通常会使用真空泵抽去其中的气体,形成真空层,从而有效避免热量损失。
	\subsubsection{技术参数}
	目前,伴随着航空航天业的发展,我国的低温容器已经有了较大的发展且制定了较为完善的体系。在GB18442-2001中,我国针对固定式真空低温绝热以及真空粉末低温绝热容器已经制定了较为完善的标准。设计储罐的过程中,需要满足以下表1的标准。
	
	\begin{figure}[H]
		\centering
		\includegraphics[width=0.9\linewidth]{biao1}
		\caption*{表2-1 立式和卧式高真空多层低温绝热容器静态蒸发率指标}
		\label{fig:biao1}
	\end{figure}
	
	\subsection{气瓶材料选择与制作}
	超低温材料的选取和使用是低温容器中十分重要的一个部分,通常而言其需要满足以下5个方面的基本要求\upcite{ref3}:
		
	(1)在工作温度范围内,膨胀系数要尽可能小
		
	(2)导热系数小
		
	(3)具有一定的低温冲击韧性,或比较低的低温塑脆转折点温度
		
	(4)对于冲击载荷以及疲劳应变有较强的适应能力
		
	(5)在低温下仍然保持较好的强度以及延展性
	\subsubsection{内筒选材及其特性}
	目前常用的低温容器选材包括9\%Ni钢、Q345R不锈钢、5083铝合金等,其中9\%Ni钢凭借着其优异的性能,发展历史较为悠久,使用的最为广泛。
		
	19世纪40年代,美国率先发现该合金,并在1952年以9\%Ni钢作为内层结构的材料制作LNG储罐,即液化天然气。在20世纪50年代中期,9\%Ni被列入ASTM标准。在20世纪79年代后期,随着液化天然气需求量的发展以及技术的进步,9\%Ni中的P、S含量分别下降到50ppm以及10ppm以下,逐渐取代了Cr-Ni不锈钢,并开始被大批量的生产应用于大型LNG储罐的制造中\upcite{ref4}。
		
	9\%Ni钢中,铝合金的含量在5\%~10\%,为中合金低碳马氏体钢,其主要的化学成分、物理性能以及力学性能如表2到表4所示。在回火前,其组织主要为索氏体+马氏体+少量奥氏体,两相区淬火+回火后产生的回转奥氏体呈弥散状,均匀的分布在板条马氏体中,这就造成其在低温状态下仍然具有较好的强度和韧性。
	\begin{figure}[H]
		\centering
		\includegraphics[width=0.6\linewidth,height=6cm]{biao2}
		\caption*{表2-2 9\%Ni钢各标准下主要成分要求}
		\label{fig:biao2-2}
	\end{figure}
	\begin{figure}[H]
		\centering
		\includegraphics[width=0.6\linewidth,height=4cm]{biao3}
		\caption*{表2-3 9\%Ni钢物理特性}
		\label{fig:biao2-3}
	\end{figure}
	\begin{figure}[H]
		\centering
		\includegraphics[width=0.6\linewidth,height=2cm]{biao4}
		\caption*{表2-4 9\%Ni钢力学性能}
		\label{fig:biao2-4}
	\end{figure}
	在实际选用中,选择购买304不锈钢,能有效满足储存液氢的需要。
	\subsubsection{焊接结构设计}
	低温容器的焊接要求非常高,这关系到容器的气密性,安全性等诸多因素,满足较高的焊接要求的容器能有效避免热量损失。
	
	低温容器焊接结构设计的关键在于焊接接头需要满足全焊接,即不能存在因焊接而出现的裂纹、气孔等人为缺陷,导致容器存在低应力脆断的风险。查阅文献\upcite{ref5},在所有内容器中接管与封头的连接均采用对接接头的形式,同时使用氩弧焊保证全焊透,使得焊缝有足够的强度和气密性。
	
	在焊接前,需要按照JB/T 4708《钢制压力容器焊接工艺评定》进行工艺评定,焊接时需要按照JB/T 4709《钢制压力容器》焊接规范进行。在焊接结束后,需要继续进行无损检测,因为在低温下奥氏体会存在向马氏体转变的现象,所以需要通过深冷处理稳定内胆的金相组织。
	
	\subsubsection{支撑材料选材及特性}
	作为低温容器的支撑材料,一方面需要满足足够的强度,能够承受内筒以及低温液体的静载荷以及作为车载容器在运输过程中产生的冲击载荷,另一方面,支撑材料处于低温与常温之间,两端温差极大,这就要求支撑材料需要满足低热导率,减小容器通过支撑件热传导导致的漏热损失。目前,支撑材料最常用的材料是玻璃纤维增强塑料,即俗称的玻璃钢\upcite{ref6,ref7}。
	
	玻璃钢是一种纤维增强复合塑料的一种,具有轻质高强、低导热系数且低温下热力性能好于常温的性能优势,在低温领域运用广泛。由于其加工工艺较为简单、材料成本较低,其应用前景也十分广泛,能有效应用于低温容器支撑材料中。
	
	
	\begin{table}[H]
		\centering
		\begin{tabular}{|c|c|c|c|c|c|c|c|c|}
			
			\hline
			\multirow{2}{*}{材料} & \multicolumn{2}{c|}{抗张强度$kg/mm^{2}$} & \multicolumn{2}{c|}{抗压强度$kg/mm^{2}$} & \multicolumn{3}{c|}{\begin{tabular}[c]{@{}c@{}}冲击强度$kg/mm^{2}$\end{tabular}} & \begin{tabular}[c]{@{}c@{}}导热系数\\ 千卡/米*时*度\\ $5\times10^{-5}$毫米汞柱\end{tabular} \\ \cline{2-9} 
			& 室温 & $77K$ & 室温 & $77K$ & 室温 & $77K$ & $20K$ & 室温$-77K$ \\ \hline
			6911 & 13.7 & 20.1 & 6.9 & 28.0 & 4.6 & 6.3 & 9.2 & 0.244 \\ \hline
			634 & 3.6 & 8.4 & 15.8 & 29.5 & 2.8 & 7.5 & 6.3 & 0.294 \\ \hline
			252-聚酰胺 & 8.6 & 15.2 & 10.9 & 28.0 & 3.3 & 6.3 & 8.0 & 0.272 \\ \hline
			618-聚酰胺 & 12.6 & 19.4 & 12.2 & 30.5 & 3.9 & 6.2 & 5.6 & 0.344 \\ \hline
			648-聚酰胺 & 7.1 & 8.2 & 6.7 & 21.2 &  & 6.1 & 6.8 & 0.246 \\ \hline
		\end{tabular}
		\caption*{表5 5种典型玻璃钢在低温下的机械性能以及导热系数\upcite{ref8}}
		\label{tab:biao5}
	\end{table}
	
	但需要注意的是,对于本文所使用的的液氢储存而言,其温度大约在20K,此时其热力学性能相较于77K及以上而言有所下降,其压缩强度如图2-1所示。在77K以上温度时,玻璃钢毫无疑问是最好的选择,但在20K左右时,可以看到碳纤维的性能更优优势。因此,对于液氢而言,可考虑同时采用碳纤维增强复合材料以及玻璃钢,充分发挥各种复合材料在不同温度下的性能优势,使得内支撑结构有效减少散热。
	
	\begin{figure}[H]
		\centering
		\includegraphics[width=0.7\linewidth]{tu2-1}
		\caption*{图2-1 各类复合材料在4-295K的性能变化}
		\label{fig:tu2-1}
	\end{figure}
	
	
	\subsection{液氢储罐结构设计}
	\subsubsection{整体尺寸设计}
	对于普通民用轿车而言,其大致尺寸与图2-2使用高压氢气储氢的丰田mirai氢气汽车相仿。
	
	\begin{figure}[H]
		\centering
		\includegraphics[width=0.7\linewidth]{tu2-2}
		\caption*{图2-2 丰田mirai}
		\label{fig:tu2-2}
	\end{figure}
	
	
	其容积为60L-70L,故可以初步设计液氢气体钢瓶大致尺寸为内筒直径300mm,外筒直径400ml,体积约为60L。
	
	\begin{figure}[H]
		\centering
		\includegraphics[width=0.7\linewidth]{biao6}
		\caption*{表6 各厚度基本描述}
		\label{fig:biao6}
	\end{figure}
	
	
	在GB 150中,对于低温气瓶的最小厚度也存在一定的要求,其命名如表6所示,互相之间的要求如图2-3所示\upcite{ref10}。
	
	\begin{figure}[H]
		\centering
		\includegraphics[width=0.7\linewidth]{tu2-3}
		\caption*{图2-3 各厚度之间的关系}
		\label{fig:tu2-3}
	\end{figure}
	
	
	基于以上描述,对内筒厚度以及内筒封头进行设计和计算。
	
	对于内筒体而言,其壁厚可以按以下公式计算
	\begin{equation*}
	\delta=\dfrac{p_{c}D_{I}}{2[\sigma]^{t}\varphi-p_{c}}
	\end{equation*}
	式中:
	
	$\delta$ \text{——计算厚度},$mm$
	
	$p_{c}$ \text{——计算压力},$MPa$
	
	$D_{I}$ \text{——筒体内径},$mm$
	
	$[\sigma]^{t}$ \text{——设计温度下的许用应力},$MPa$
	
	$\varphi$ \text{——焊接接头系数}
	
	其中,由于是完全焊接,焊接接头系数为1,其余数据均可以通过查询相关标准得到。
	
	结合文献\upcite{ref11}以及施工工艺要求可以得到最终的名义壁厚$\delta_{n}=5mm$
	
	对于封头而言,以凸形封头运用最为广泛,参考文献,选取DHA蝶形封头,利用公式
	\begin{equation*}
	\delta=\dfrac{Kp_{c}D_{I}}{2[\sigma]^{t}\varphi-0.5p_{c}}
	\end{equation*}
	
	计算可以得到封头名义厚度同样为$\delta_{h}=5mm$
	
	最后查表得到封头体积为$V_{\text{封}}=0.0053m^{3}$
	
	计算筒体长度
	\begin{equation*}
	L=\dfrac{V-2V_{\text{封}}}{\pi r^{2}}
	\end{equation*}
	
	得到筒体长度$L=730mm$
	
	同理,对于外筒同样进行相似的计算,可以得到外筒的名义壁厚同样为$\delta_{n}=5mm$,筒体长度$L=840mm$
	\subsubsection{支撑结构设计}
	作为漏热的重要部分,文献表明其漏热占比相当高\upcite{ref12},其支撑结构占总体漏热的36.3\%,对整体漏热影响巨大,因此一个合理的支撑结构设计很大程度上影响了容器的整体漏热。
	
	目前低温系统中广泛使用的有两种支撑形式,分别为支撑管/柱以及吊拉带,其中支撑管主要承受压缩载荷,吊拉带主要承受拉伸载荷,其内支撑结构形式如图2-4所示\upcite{ref13}。受拉伸的构件两固定端应留有一定的活动余隙,否则由于内胆的冷收缩拉杆受力太大,会在两固定端产生很大应力。内支撑结构通常使用的材料为上文介绍过的玻璃钢,凭借其优秀的性能有效减小漏热的情况。支撑管主要均匀分布在内外容器之间,对强度和载荷均有一定的要求和限制。
	\begin{figure}[H]
		\centering
		\includegraphics[width=0.7\linewidth]{tu2-4}
		\caption*{图2-4 低温容器内支撑形式}
		\label{fig:tu2-4}
	\end{figure}
	
	
	\subsubsection{测量结构设计}
	对于低温液体,出于需要了解容器内液面高度以及容器内液体状态的需要,我们需要对内瓶内的液面进行测量。常用的测量方法包括差压式液面计,其简要原理图如图2-5所示。
	\begin{figure}[H]
		\centering
		\includegraphics[width=0.7\linewidth]{tu2-61}
		\caption*{图2-5 压差式液面计原理图}
		\label{fig:tu2-5}
	\end{figure}
	
	其主要原理如下:液面计的液相端与内胆底部相连,气相端与内胆顶部相连。当液面计的液相阀与气相阀打开而平衡阀关闭,如果此时容器内装有一定量的低温液体,液面计就能反映出这种状态。当容器内全部是气体时,内胆顶部与底部压力相等,即液相端和气相端相等。液面计能准确反映容器底部与顶部空气之间的压力差,通过压力差进行换算$\Delta p=p-p_{0}=\rho gh$或者是查表等就可以准确得到液面的高度\upcite{ref9}。
	
	\subsubsection{增压管及排液管设计}
	对于低温液体而言,将液体排出容器常用的方法有三类,分别是:内胆的自增压、外部气体对容器内胆加压以及液泵输送。这里我们选择通过增压管对内胆进行自增压排液,其示意图如图2-6所示,主要部件分别是:1、外夹套 2、增压加气管 3、内筒 4、增压管 5、增压器。其主要原理是利用低温液体在增压器中被加热转化为低温气体,继而通过回气管进入内筒空间,逐步提高内筒压力,从而达到增压的效果\upcite{ref14}。
	
	对于排液管而言,通常会选择结构简单、加工技术成熟、不易结垢且综合换热性能较好的波纹管,在低温下也有着较为优秀的性能。
	
	\begin{figure}[H]
		\centering
		\includegraphics[width=0.7\linewidth]{tu2-7}
		\caption*{图2-6 低温容器内增压管路}
		\label{fig:tu2-6}
	\end{figure}
	
	\section{液氢气瓶的绝热分析}
	\subsection{绝热结构的分类}
	对于低温容器,绝热结构的确定是其设计的主要工作之一。为达到绝热最终目标,首先可采用不良导热体以减弱固体接触传热,其次可通过真空或粉末绝热以抑制夹层空间内的气体分子传热,再次可采用金属镀层和热屏蔽层以减小空间内的辐射传热等方法。
	\subsubsection{普通堆积绝热}
	具体结构是在需要绝热的容器外面建造一个夹层空间,在夹层空间里装填或包覆一定厚度的绝热材料,在夹层空间进行抽真空工艺,最终达到绝热目的,目前在大型空分制冷装置、LNG贮存装置等方面各得到广泛的应用,一些特大型容积超过1000m³需现场制作的液氢贮槽及试验设备中一般也采用此种结构。隔热材料常用固体泡沫型、粉末和纤维型等类型。普通堆积绝热成本较低,无需真空罩,常用于不规则形状的容器加工或超大容积需现场施工的低温工程项目,但罐体真空度不高绝热性能较差。
	\subsubsection{真空绝热}
	真空绝热方式是在一个密闭空间内保持真空的绝热形式,按夹层空间内绝热材料使用的不同可分为\textbf{真空多孔材料绝热、高真空复合多层缠绕绝热和高真空单层多层缠绕绝热}等。高真空绝热亦称单纯真空绝热,内外容器夹层空间真空度一般可以维持在$1.33×10^{-3}Pa$,这时影响低温区的热量主要来源于夹层空间里的辐射传热和小量残余气体的传导热以及固体构件的传导热。高真空度绝热层具有结构简单、紧凑、热容量小等优点,适用于小型液化天然气贮存、少量液氧、液氮、液氩以及少量短期的液氢贮存,由于高真空度的获得和保持比较困难,需要耗费大量的能源用真空泵来抽真空,所以一般在大型贮罐中很少用。真空多孔材料绝热是在内外容器夹层空间内填充多孔性绝热材料如多孔粉末或多孔纤维,然后用机械抽真空泵将夹层空间的真空度抽到10Pa以下,消除多孔材料气体的对流传热,真空多孔材料绝热结构所维持的真空度不高,但是其真空绝热性能比普通堆积绝热要高两个数量级,因此广泛用于大、中型低温液体贮存中,如液化天然气贮存、液氧、液氮运输设备及量大的液氢船运设备中,其最大的缺点是要求内外容器间的夹层空间要足够的大,这就造成了整个低温储罐结构复杂,外容器体积较大。多层绝热又称高真空复合多层绝热,是一种在内外容器绝热真空空间中缠绕一定层数平行于低温冷屏的辐射屏和复合具有低热导率的间隔物形成的高效绝热结构,绝热夹层空间的真空度可以达到或优于10-2Pa,辐射屏材料常采用铝箔、铜箔或喷铝涤纶薄膜等具有高辐射能力的材料,间隔物材料常用玻璃纤维纸或植物纤维纸、尼龙布、涤纶膜等,使绝热层中辐射、固体导热以及残余气体热导都减少到了最低程度,绝热性能卓越,因而亦被称为“超级绝热” \upcite{ref15}。真空多层绝热结构特点是绝热性能卓越,重量轻,预冷损失小,但制造成本高,抽空工艺复杂,难以对复杂形状绝热,应用于液氧、液氮的长期贮存,液氢、液氦的长期贮存及运输设备中。高真空多屏绝热是一种多层缠绕绝热与蒸气冷却屏相结合的绝热结构形式,在多层绝热中采用由挥发蒸气冷却的汽冷屏作为绝热层的中间屏,由挥发的蒸气带走部分传入的热量,以有效地抑制热量从环境对低温液体的传入。多屏绝热是多层绝热的一大改进,绝热性能十分优越,热容量小、质量轻、热平衡快,但结构复杂,成本高,一般适用于液氢、液氦的小量贮存容器中。
	
	对于我们小组的移动式车载液氢气瓶,\textbf{我们选用的是高真空复合多层绝热结构},主要原因在于:一是考虑在液氢温区,该结构既能满足绝热要求,同时结构又相对简单。二是对于移动式压力容器,采用该结构可以减少绝热层的重量,能够有效提高底盘限载情况下的有效载液量。根据真空手册计算真空夹层之间的漏热量,辐射传热约占总漏热量的25\%,多层缠绕材料之间的固体传导传热约占总漏热量的5\%,绝热真空空间的残余气体导热约占总漏热量的70\%。影响液氢储罐绝热性能的因素有多层材料的种类及其组合方式、多层绝热材料的层密度厚度、绝热夹套中的真空度以及绝热内外壁的边界温度等。
	\subsection{高真空多层绝热材料选择及其绝热性能影响因素}
	高真空多层绝热材料结构如图3-1所示,它由许多辐射屏与间隔物交替组成。具有高反射能力的辐射屏大幅减少辐射热,为了阻止辐射屏的接触产生热短路,屏间加入热导率很小的材料绝热空间被抽到优于10-2的真空度,并加入吸附剂保持其真空度。辐射屏的材料常用铝箔、铜箔或喷铝涤纶薄膜等,尤其以价廉而且轻便的镀铝薄膜和铝箔最为常见;间隔物材料通常采用玻璃纤维纸、植物纤维纸、化纤纸、丝绸等绝缘材料。影响其传热的有如下几方面因素:反射屏和间隔物材料的种类和组合方式,绝热空间的真空度,绝热层的层密度,多层的层数,温度,包扎的松紧度等。
	\begin{figure}[H]
		\centering
		\includegraphics[width=0.2\linewidth]{tu3-1}
		\caption*{图3-1 高真空多层绝热材料结构简图}
		\label{fig:tu3-1}
	\end{figure}
	\subsubsection{多层绝热材料的选择}
	为有效减少辐射热,辐射屏的表面应是光洁的,具有高反射率和低发射率,金属材料满足上述要求。其中,铝价格较低,也不易氧化是最为合适的。铝的热导率较高,反射屏应尽量薄,以增加反射屏的层数。而很薄的金属箔强度低,易破坏。将金属镀在薄膜上能够解决上述矛盾,且使多层材料更轻。高真空多层材料中的间隔物,不但具有隔热的作用,而且对辐射波有一定的吸收和散射作用。热导率低、结构疏松的材料能够很好的满足要求。对于玻璃纤维纸的实验研究表明,间隔物与反射屏之间的接触热阻大于间隔物自身的热阻。因此间隔物一般为网状,以减小屏与隔绝物之间的接触面积。
	\subsubsection{绝热空间的真空度}
	多层绝热材料的真空度可表述为两种:间壁空间内的真空度被称之为表观真空度和多层绝热材料内部的真空度。多层材料会明显影响抽真空的速率,层间气体会不容易抽走,表观真空度与材料内的真空度会显著不同。为尽可能减小残余气体导热对多层绝热材料的影响,真空度应优于$10^{-2}Pa$。为了保证绝热空间内的真空度满足要求目前采用较多的方案有:加热抽空(一般小于125℃)、惰性气体置换(如氮气)、在绝热空间内加入吸附剂、采用易于抽空包扎方式和在屏上开孔等。
	\subsubsection{层密度}
	复合多层绝热材料单位厚度中辐射屏的层数称为层密度。复合多层绝热材料的层密度受选用材料、包扎的松紧程度等因素的影响。多层材料的表观热导率与损失热流量随层密度变化曲线如图3-2所示。
	\begin{figure}[H]
		\centering
		\includegraphics[width=0.6\linewidth]{tu3-2}
		\caption*{图3-2 表观热导率与损失热流量随层密度变化曲线}
		\label{fig:tu3-2}
	\end{figure}
	综上,我们选择的是表观导热系数小的\textbf{玻璃纤维纸和厚度为0.065mm铝箔的组合},考虑真空度的影响因素,要求封口真空度不小于$10^{-3}Pa$,同时为了保证抽气效果,绝热层上按工艺要求进行打孔,以利于抽气及提高真空度;压缩载荷和层密度在工艺文件中有明确要求,针对玻璃纤维纸和铝箔的组合,工艺规定\textbf{层密度为25层/cm}。
	\subsection{液氢贮罐漏热蒸发率计算}
	衡量绝热性能的最重要的参数是低温容器的蒸发率指标, 所谓低温容器的蒸发率\upcite{ref16},是指在标准状态下(101.3kpa,0℃),采用低温容器储存适量的低温液体,在罐内达到热平衡后的蒸发速率。一般以24小时计算,故又称日蒸发率。具体量化计算公式是指24小时内蒸发的液体质量与该低温容器的额定装载量之比:
	\begin{equation*}
		\eta=\dfrac{M}{V_{b}}\times 100\%
	\end{equation*}
	
	式中,$\eta$为日蒸发率;$V_{b}$为低温容器的额定装载量;$M$为24小时内蒸发的液体量。
	\subsubsection{高真空多层绝热漏热}
	朱公先\upcite{ref17}引入残余气体对于传热的影响,对前期传热模型进行了修正,采用能量守恒作为基本方法推导出通过高真空多层绝热材料比热流的半经验计算公式,同时对反射屏为双面镀铝薄膜,相邻两层反射屏之间的隔绝物为两层丝网织物的多层材料进行了分析,获得如下经验公式
	\begin{equation*}
	q=\dfrac{C_{r}\varepsilon_{RT}}{N_{S}}(T^{4.67}_{H}-T^{4.67}_{C})+\dfrac{C_{s}(n)^{2.56}T_{m}}{N_{S}+1}(T_{H}-T_{C})
	\end{equation*}
	式中:$q$——通过多层的比热流;
	
	$C_{r}=5.39\times10^{-10}$;
	
	$C_{s}=8.59*10^{-8}$;
	
	$n$——层密度;
	
	$N_{S}$——反射屏数;
	
	$T_{H}$——热边界温度;
	
	$T_{C}$——冷边界温度;
	
	$T_{m}$——定性温度;
	
	$\varepsilon_{RT}$——室温300K的表面发射率。
	
	由上式可以计算出多层绝热漏热的比热流,取双面镀铝薄膜发射率0.05,$q_{1}=0.23W/m^{2}$。绝热材料面积取内容器外表面积,$A=0.688m^{2}$,漏热量为$Q_{1}=A\cdot q_{1}=0.15824W$.
	\subsubsection{内外筒连接支撑漏热量计算}
	支撑管漏热量按下式计算\upcite{ref18}:
	\begin{equation*}
	Q_{2}=\dfrac{A}{L}
	\left[\int^{T_{2}}_{4}\lambda(T)\mathrm{d}T-\int^{T_{1}}_{4}\lambda(T)\mathrm{d}T\right]=1.099W
	\end{equation*}
	式中:
	$A$——接管的横截面积,取$0.94cm^2$;
	
	$L$——接管的长度,取$30cm$;
	
	$T_{2}$、$T_{1}$分别为吊杆两端的温度,取$T_{1}=20K$、$T_{2}=300K$。
	
	综上,气罐总漏热量为$Q=Q_{1}+Q_{2}=1.25629W$:
	
	取安全系数$K=1.2$,则$Q^{'}=53.001W$
	
	液氢的日蒸发率$\eta$为
	\begin{equation*}
	\eta=\left[\dfrac{24\times3600\times Q}{r\times\rho\times V}\right]\times 100 \% =6.53\%
	\end{equation*}
	式中,$r$——液氢在$20K$时的汽化潜热,为$460.5KJ/kg$;
	
	$\rho$——液氢的密度,为$70kg/m^{3}$;
	
	$V$——气瓶的容积,为$0.0516m^{3}$。
	
	\section{气瓶有限元分析}
	\subsection{模型建立}
	为了对液氢储罐进行有限元分析,首先需要将其模型建立起来。出于计算方便等原因,对模型进行了一定程度的简化。在建立的模型中仅包含内筒、外筒和支撑板,线稿图如图4-1所示。
	\begin{figure}[H]
		\centering
		\includegraphics[width=\linewidth]{tu4-1}
		\caption*{图4-1 液氢储罐简化模型线稿图}
		\label{fig:tu4-1}
	\end{figure}
	其内筒包含封头、筒体和管道,外筒则包括封头和筒体。筒体和管道的模型为简单圆柱体,封头的模型较为复杂,选择从网上获取已有的模型。具体而言,内筒的长度为730mm,外筒长度为840mm。内筒的封头选择直径300mm、名义厚度5mm的DHA蝶形封头\upcite{ref19},外筒的封头选择直径400mm、名义厚度5mm的DHA蝶形封头\upcite{ref20}。支撑板厚度为10mm。具体参数工程图见附页。
	再将模型导入SpaceClaim进行进一步的处理,主要目的是将其转化成能使用ANSYS处理的格式。
	\subsection{网格划分}
	网格的划分是有限元分析处理的重要部分,很多情况下网格划分的好坏直接决定着最后分析结果的正确性。但由于对这方面的知识了解不多,选择使用Ansys Mechanical的自动网格划分来划分网格。考虑到精度和计算速度,对参数进行了一定的调整,将网格的大小设置在1cm,最后画了大约22万个网格。划分网格后的模型如图4-2所示。
	
	从图中可以看出在支撑板附近的网格还是有点稀疏,因为支撑板的厚度也只有1cm,和网格大小相当,使得整个支撑板只由一层网格组成,是不太合理的。
	\begin{figure}[H]
		\centering
		\includegraphics[width=\linewidth]{tu4-2}
		\caption*{图4-2 液氢储罐网格图}
		\label{fig:tu4-2}
	\end{figure}
	\subsection{条件设置}
	为了进行有限元分析,需要给模型加上一些初始条件,如模型各部分的材料、承受压力、约束等。
	\subsubsection{材料条件设置}
	在材料的设置上,为外筒体选择了常温下的不锈钢对应的数据,数据来源为ANSYS自带的数据库。而考虑到材料性质在低温时会有较大的变化,为内筒体选取了温度在氢气沸点附近(20K)对应的不锈钢材料,数据来源为NIST数据库\upcite{ref21}。由于不锈钢在20K温度对应性质数据较少,泊松比和密度数据选择和常温一致的数据填入。性质具体数据值见图4-3。
	\begin{figure}[H]
		\centering
		\includegraphics[width=\linewidth]{tu4-3}
		\caption*{图4-3 不锈钢常温数据}
		\label{fig:tu4-3}
	\end{figure}
	\begin{figure}[H]
		\centering
		\includegraphics[width=\linewidth]{tu4-4}
		\caption*{图4-4 不锈钢20K数据}
		\label{fig:tu4-4}
	\end{figure}
	\subsubsection{约束设置}
	在z轴负方向上设置了重力加速度g;在内筒的内壁上设置了1.6MPa的压力。在外筒的外壁上设置了正常的大气压。为了将容器固定住,在外筒的外壁上施加了两条环向约束。具体如图所示。A为外界大气压,B为内壁压力,C为重力加速度,D为固定约束。
	\begin{figure}[H]
		\centering
		\includegraphics[width=0.8\linewidth]{tu4-5}
		\caption*{图4-5 约束条件}
		\label{fig:tu4-5}
	\end{figure}	
	\subsection{求解}
	设定好条件后就可以在软件中进行求解了。Ansys Mechanical提供了多种求解功能,但考虑到初始条件和实际需求,选择了应力和变形进行分析。
	首先是对应力的分析,使用的方法是von Mises准则。加入了一个剖面图来观察筒体内部应力,得到的应力云图如图所示。最大的应力约为114MPa。可以发现最大的应力集中在液体流出管道和外筒连接的部分,支撑板和内筒连接部分也承受了较大的应力。
	\begin{figure}[H]
		\centering
		\includegraphics[width=0.5\linewidth]{tu4-6a}
		\caption*{图4-6(a)管口应力}
		\label{fig:tu4-6a}
	\end{figure}
	\begin{figure}[H]
		\centering
		\includegraphics[width=0.8\linewidth]{tu4-6b}
		\caption*{图4-6(b)应力剖面图}
		\label{fig:tu4-6b}
	\end{figure}
	接下来是对筒体的变形分析。云图如图所示。大致趋势为外筒向内收缩,内筒向外膨胀。最大的变形集中在内筒的尾部,约为0.016\%,可见对内筒在尾部的支撑也是不能忽略的。
	\begin{figure}[H]
		\centering
		\includegraphics[width=0.7\linewidth]{tu4-7}
		\caption*{图4-7 变形云图}
		\label{fig:tu4-7}
	\end{figure}
	\subsection{一些其他的分析}
	该部分包括一些失败的分析与思考。
	
	对模型进行了热分析,但只是简单给内外筒加上了不同的温度条件,即20k和300k。然后得到了难以想象的结果——模型的内筒发生了巨大的形变,甚至使其超出了外筒,情况如图。这样的结果是不合理的,推测可能的原因有以下几个:
	
	1.材料物性数据有误:所使用的材料物性参数不准确,导致了错误的结果。
	
	2.模型设置不当:在模型建立过程中简化了不能忽略的条件。
	
	3.使用了错误的约束条件:如对温度的设定。在软件中温度的设定是具体到body,即单个体上的,但实际上不太可能保证单个部分的温度完全一致。
	
	4.选用的计算方法不适合此条件:应当考虑其他的计算方法。
	\begin{figure}[H]
		\centering
		\includegraphics[width=\linewidth]{tu4-8}
		\caption*{图4-8加入温度条件后的应变云图}
		\label{fig:tu4-8}
	\end{figure}
	在本次的分析中还尝试了参数化设计方法分析支撑环位置对结构强度影响。即将支撑环位置作为变量输入模型,通过对变量值取不同值来获得不同的结果,以确定最优设计,但由于一些问题未能成功进行。尝试了使用fluent进行传热的分析,但因为网格划分不正确,无法导入进行分析。
	
	在进行展示后,老师提出了很多很好的意见,但由于时间和考试等原因,要进行改进较为困难。我在本次项目主要的收获是大致了解了有限元分析相关的知识和软件的使用,为以后的学习打下了一定的基础。同时也对在低温技术与应用课程学习到的知识进行了一定程度的应用。
	\newpage
	\clearpage
	\begin{thebibliography}{99}
		
		\phantomsection
		\addcontentsline{toc}{section}{References}
		\bibitem{ref1}Yamane K et al.A Study on the Effect of the Total Weight of Fuel and Fuel Tank on the Driving Performances of Cars,Int.J.Hydrogen Energy, 1998,(23)9
		\bibitem{ref2}梁焱,王焱,郭有仪,白春泽.氢动力车用液氢贮罐的发展现状及展望[J].低温工程,2001(05):31-36.
		\bibitem{ref3}华学明,蔡艳,吴毅雄,王欢,石少坚,唐永生.大型LNG船围护系统低温金属材料焊接技术现状及发展[J].电焊机,2015,45(05):28-35.
		\bibitem{ref4}白建斌,董海青,朱海滨,曲维春,李佳恒,李鑫利,刘洋.9Ni钢及其焊材发展现状[J].电焊机,2021,51(04):57-61+11.
		\bibitem{ref5}路兰卿,于洋.高压液氢容器的研制[J].航天制造技术,2013(03):50-52+64.
		\bibitem{ref6}杨超. 低温容器用玻璃钢支撑结构的接触导热性能和工程应用[D].北京工业大学,2016.
		\bibitem{ref7}路兰卿,于洋.高压液氢容器的研制[J].航天制造技术,2013(03):50-52+64.
		\bibitem{ref8}黄家康,张春堂.玻璃钢的低温性能与应用[J].低温物理,1979(03):247-251.
		\bibitem{ref9}刘东进,顾华.船用LNG储罐液位测量装置[J].辽宁化工,2020,49(10):1297-1299.
		\bibitem{ref10}杜明广. 车载LNG气瓶的设计与分析[D].合肥工业大学,2016.
		\bibitem{ref11}李万晖. 车载LNG气瓶设计及ANSYS分析[D].青岛科技大学,2017.
		\bibitem{ref12}夏园园,赵腾飞,段武,刘蓓蓓,余谦.设计LNG气瓶时降低静态蒸发率理论计算与实际的试验研究[J].化工管理,2017(24):29-30.
		\bibitem{ref13}赵福祥,魏蔚,刘康,汪荣顺.纤维复合材料在低温容器内支撑结构中的应用[J].低温工程,2005(03):23-26+34.
		\bibitem{ref14}王明富,王立.航天发射场低温容器管道设计[J].真空与低温,2016,22(01):47-52.
		\bibitem{ref15}Sherif SA, Zeytinog˘lu N, Vezirog˘lu N. Liquid hydrogen: potential, problems and  a  proposed  research  program[J].  Int  J  Hydrogen  Energy,  1997,  22:683–688.
		\bibitem{ref16}GB/T 18442.1~GB/T 18442.6-2011,固定式真空绝热深冷压力容器[S].
		\bibitem{ref17}朱公先. 多层绝热中一个新的半经验公式的建立[J].  深冷技术, 1983,(2).
		\bibitem{ref18}王如竹,汪荣顺. 低温系统[M].上海:上海交通大学出版社,2000:241-256.
		\bibitem{ref19}CIDP,海枣数字-DHA蝶形封头DN300(JB/T 4746—2002)[EB/OL],2021-05-09,\url{https://www.digitalmechanical.com.cn/web/Part/babc7423f6604f04a1163b4e28186603}%{https://www.digitalmechanical.com.cn/web/Part/babc7423f6604f04a1163b4e28186603}
		\bibitem{ref20}CIDP,海枣数字-DHA蝶形封头DN400(JB/T 4746—2002)[EB/OL],2021-05-09,\url{https://www.digitalmechanical.com.cn/web/Part/a8b75de86c3c4b9ba752751f9a931e37}%{https://www.digitalmechanical.com.cn/web/Part/a8b75de86c3c4b9ba752751f9a931e37}
		\bibitem{ref21}NIST,cryogenic material properties 304 Stainless[EB/OL], 2021-05-09,\url{https://trc.nist.gov/cryogenics/materials/304Stainless/304Stainless_rev.htm}%{https://trc.nist.gov/cryogenics/materials/304Stainless/304Stainless\underline{~}rev.htm}
		
	\end{thebibliography}
	\newpage
	\section*{签名页}
\end{document}